\chapter{Conclusion}

Ce Projet Personnel des Humanités, avant d'\^etre un travail à rendre, s'est inscrit dans la continuité d'un questionnement acquis lors de mon engagement dans l'association HandiManagement et des rencontres humaines fortes que j'ai pu y faire.\\

J'avais le sentiment que les entreprises rencontrées lors de la période d'acculturation, bénéficiaient d'un climat social de qualité. Au cours de mes recherches progressives sur les difficultés d'insertion professionnelles des personnes en situation d'handicap, j'ai pris conscience que ce n'étaient pas les lois qui créaient les différences, mais que c'était les Hommes qui construisaient les barrières.\\

Les rencontres et les données de la littérature m'ont permis de mesurer tout le travail qu'il reste à faire pour amener la personne en situation de handicap vers l'emploi.\\
Des actions de sensibilisation, des organismes d'accompagnement, un engagement plus actif de l'Agefiph pour sensibiliser et encourager les entreprises, et une véritable volonté politique des pouvoirs publics pour offrir à la personne handicapée une place dans la société, sont autant de facteurs nécessaires pour augmenter l'employabilité des personnes marquées par un handicap acquis, m\^eme dans une conjoncture défavorable pour l'emploi.