\chapter{Difficultés liées à l'entreprise}

Chiffre du chômage pour les personnes handicapées ?

Les difficultés d'insertion professionnelle des personnes handicapées en entreprises dépendent de nombreux facteurs. Certains sont professionnels (manque de qualifications pour l'obtention du poste) quand d'autres se basent sur des critères socio-culturels (préjugés ou peur du handicap).

PME ou grosses entreprises ? Entreprises publiques ou privées ?

\section{Manque de qualifications}

Une des principales raisons de non-insertion des personnes handicapées est le manque de qualifications de la personne handicapée face au travail. Cette conclusion se remarque par exemple dans des écoles d'ingénieurs : à l'INSA de Lyon, sur près de 5000 étudiants, il n'y a que 57 personnes en situation d'handicap.\\

Beaucoup d'entreprises sont à la recherche de salariés bénéficiant d'un haut niveau de compétences (Thales ou Renault Trucks par exemple), c'est à dire minimum BAC + 5.

Si l'on observe la répartition du niveau scolaire des personnes handicapées recherchant un emploi, on s'aperçoit vite que la plupart des personnes ne disposent pas d'un diplôme élevé.
En se basant sur le rapport d'activité 2010 de l'association Comète France, on peut en effet observer que près de 47 \% des personnes handicapées recherchant un emploi ne disposent seulement que d'un BEP ou d'un CAP.\\

\begin{figure}[h!]
\centering
\includegraphics[scale=0.5]{../Repartition_Niveau_Scolaire_Personnes_H.png}
\caption{Répartition des personnes handicapées en fonction de leur niveau scolaire sur un échantillon - \textit{Comète France 2010}}
\end{figure}

Sur ce même tableau, nous observons qu'il n'y a que 8 \% des personnes handicapées qui disposent d'un niveau Licence et plus. Il est clair que ces personnes n'auront pas beaucoup de difficultés pour trouver un emploi. \\

Cependant, les 92 \% autres personnes qui disposent d'un diplôme inférieur à une licence vont rencontrer beaucoup plus de difficultés. Cela peut être dû au manque d'investissement dont lequel investisse les écoles et établissements scolaires dans l'éducation des personnes en situation d'handicap, notamment pour s'adapter à chaque type de handicap.

Ces difficultés vont également être dûs aux barrières socio-culturelle que je vais détailler dans les prochaines parties.


\section{Barrière socio-culturelle}

\subsection{Stéréotypes et Préjugés Négatifs}
Un stéréotype est une idée, une opinion, acceptée sans réflexion et répétée, sans avoir été soumise à un examen critique, par un personne ou par un groupe, et qui détermine, à un degré plus ou moins élevé, ses manières de penser, de sentir et d'agir.
La conséquence des stéréotypes est une généralisation abusive d'une idée. 
Parmi un témoignage, un chef d'une entreprise industrielle du secteur de l'énergie pensait que le recrutement de personnes handicapées n'était pas envisageable parce que son bâtiment ne possédait pas d'escaliers. Il partait donc sur le stéréotype qu'une personne handicapée est forcément en fauteuil roulant. Or parmi l'ensemble des personnes handicapées, près de 3 \% sont effectivement en fauteuil roulant. Ce responsable avait donc développé un stéréotype basé sur le fait qu'il ne pourrait jamais embaucher de personnes handicapées. Alors qu'il pouvait embaucher près de 97 \% de personnes handicapées pour lesquelles prendre les escaliers n'était pas un problème.\\

Plus généralement, on remarque que le stéréotype le plus commun pour le milieu du handicap est de croire que le handicap est toujours visible. Or, près de 80 \% des personnes handicapées ont un handicap invisible. Les handicaps psychiques (dépression, schyzophrénie, etc.), cognitifs (dyslexie...), mentaux ou sensoriels sont majoritairement des handicaps invisibles.\\

Un préjugé signifie "juger avant". Il est défini comme une opinion hâtive et préconçue souvent imposée par le milieu, l'époque, l'éducation ou due à la généralisation d'une expérience personnelle ou d'un cas particulier. Par exemple, nos comportements envers une personne dépendent souvent de ce que nous imaginons de elle, et pas de ce qu'elle est réellement.\\
Les effets des stéréotypes et de la généralisation étant souvent plus importants sur les populations minoritaires, si le recrutement d'une personne handicapée se solde par un échec, les stéréotypes négatifs risquent d'être renforcés. A contrario, si le recrutement d'une personne non handicapée se déroule mal, alors le phénomène de généralisation n'aura pas lieu.\\
Pour prendre un exemple, on entend souvent des chefs d'entreprise dire que si l'embauche d'une personne handicapée s'est mal passée, alors le chef d'entreprise ne reprendra plus d'autres personnes handicapées. Alors qu'avec des personnes non handicapées, cette réflexion ne sera jamais entendu.\\

En 2003, une enquête a été faite par Jean-François Amadieu, professeur à l'université Paris 1 et directeur de l'Observatoire des discriminations. L'enquête consistait à envoyer à 258 entreprises, près de 1800 candidatures qui correspondaient à 7 candidats virtuels :
\begin{enumerate}
\item Un homme dit "standard" (nom et prénom français, résident à Paris, blanc de peau, apparence standard) et considéré comme le candidat de "référence".
\item Une femme dit "standard" (nom et prénom français, résident à Paris, blanche de peau, apparence standard)
\item Un homme d'origine maghrébine (nom et prénom maghrébins, résident à Paris, apparence standard)
\item Un homme résident au Val Fourré à Mantes-la-Jolie (nom et prénom français, résident à Paris, apparence standard)
\item Un homme au visage disgracieux (nom et prénom français, résident à Paris, blanc de peau)
\item Un homme de 50 ans (nom et prénom français, résident à Paris, blanc de peau, apparence standard)
\item Un homme handicapé (nom et prénom français, résident à Paris, blanc de peau, apparence standard)
\end{enumerate}

Le candidat de référence obtint près de 75 convocations.
Et outre les quelques candidats virtuels présents pour tester d'autres discriminations (\^age, racisme, etc.), le candidat qui avait déclaré son handicap reçut 5 convocations seulement.\\

Lorsque cette étude a été refaite en 2006, les résultats furent un peu plus positifs en faveur de la personne handicapée, notamment gr\^ace à la loi de 2005.\\

Ces enqu\^etes montrent à quel point l'entreprise peine à assimiler la culture de la diversité malgré les efforts réels entrepris ces dernières années. \\
En effet, les candidats étaient à compétences égales sur les curriculum vitae. C'est donc le choix des recruteurs, motivé par des préjugés négatifs, qui s'est porté plut\^ot sur le candidat de référence. La différence est en effet ressentie en entreprise comme une prise de risque dans un environnement où on se soucie plutôt de les réduire.

\subsection{Stéréotypes positifs}




\subsection{Image physique importante}
De la m\^eme façon, la non-insertion professionnelle peut \^etre d\^u à une peur de donner une mauvaise image au consommateur. Les intervenants de la cellule Comète que j'ai rencontré à Montpellier m'ont expliqué qu'il y a quelques années, ils avaient dû faire une intervention auprès de la société Loréal pour un cas de maintien dans l'emploi.
Une jeune fille avait eu un accident de la route et avait été défiguré. A cette époque, ils avaient alors décidé de la licencier car sa défiguration n'était pas en adéquation avec les idées données par la société (beauté du visage).\\

Une autre histoire concernait le secteur bancaire. Pour inspirer confiance aux clients dans la banque, il avait alors été demandé de ne pas embaucher des personnes en situation d'handicap physique, signe de faiblesse pour la banque.

\section{Discrimination positive}




\section{Mauvaise communication}