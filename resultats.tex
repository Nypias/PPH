\chapter{Résultats}
\label{resultats}

Une étude commanditée par l'Agefiph en 2009~\cite{etudeAgefiph1}, présentant des chiffres quant à l'emploi direct des personnes handicapées, souligne que les freins à l'emploi en France peuvent \^etre de 3 natures :
\begin{itemize}
\item Une inadéquation des compétences des travailleurs handicapés avec les postes à pourvoir.
\item Une méconnaissance du handicap et des types d'handicap qui crée des stéréotypes et préjugés.
\item Une méconnaissance des dispositifs d'aide proposés par l'Agefiph (conseils, aides financières, tutorat). \\
\end{itemize}

A ces trois freins, peuvent être ajoutés deux facteurs de résistance portés par la personne handicapée :
\begin{itemize}
\item Une méconnaissance des dispositifs d'accompagnement à l'insertion professionnelle
\item Le rôle paradoxal des allocations d'aides pour la personne handicapée.\\
\end{itemize}

%Dans cette partie, je me propose donc de décrire les principaux résultats trouvés qui explique un taux de chômage élevé (près de 20 \%) dans la population handicapée.



\section{Inadéquation des compétences des personnes handicapées avec les postes à pourvoir}

\subsection{Des diplômes insuffisamment élevés}
Si l'on observe la répartition du niveau scolaire des personnes handicapées recherchant un emploi, on s'aperçoit qu'une grande majorité des personnes handicapées ne disposent pas d'un diplôme élevé.
En se basant sur le rapport d'activité 2010 de l'association Comète France, on peut en effet observer que près de 84 \% des personnes handicapées recherchant un emploi disposent d'une formation inférieure ou égale au baccaulauréat.

\begin{figure}[H]
\centering
\includegraphics[scale=0.5]{../Repartition_Niveau_Scolaire_Personnes_H.png}
\caption{Répartition des personnes handicapées en fonction de leur niveau scolaire sur un échantillon - \textit{Comète France 2010}}
\end{figure}

Par ailleurs, de nombreuses entreprises se recentrent sur leur coeur de métier et sous-traitent alors les emplois à basse qualification. Ces entreprises recrutent alors des postes nécessitant un bac + 2, voire davantage.\\

Au vu des chiffres ci-dessus, il apparait alors que le niveau des compétences des personnes handicapées, en général, ne suffit pas à pourvoir des postes à haute technicité.\\
CapEmploi fait remarquer dans le rapport Dares que beaucoup d'entreprises ont une grande volonté d'embaucher des personnes handicapées et qu'elles ne peuvent pas se le permettre car le niveau de compétences requis par ces entreprises dépasse largement le niveau de compétences de cette population handicapée.\\

L'inadéquation des compétences n'est pas seulement liée aux diplômes mais à l'impossibilité d'utiliser ces diplômes et le savoir acquis avant la maladie ou l'accident. C'est le cas des personnes victimes d'une lésion cérébrale (traumatisme crânien, accident vasculaire cérébral, etc.). Les troubles cognitifs (mémoire, raisonnement, attention/concentration etc.) leur font perdre leur capacités antérieures. \\

L'inadaptation aux postes provient également d'un problème de formation. Par exemple, beaucoup de personnes handicapées postulent en tant qu'\textit{employés de bureau}. En effet, lors des HandiCafés\footnote{\url{http://www.ladapt.net/ewb_pages/h/handicafe.php}} se tenant au Forum Rhône-Alpes à Lyon, en mars 2012, un grand nombre de personnes handicapées étaient présentes pour obtenir des postes bureautiques.\\
Pourtant, les entreprises dénoncent ces formations qui n'ont plus vraiment d'intérêt pour elles car ces postes n'existent plus dans la grande majorité des entreprises.\\

Certaines entreprises évoquent enfin le défaut de choix de la formation par les personnes handicapées.
Dans le rapport DARES~\cite{etudeDares1}, une coordinatrice Apprentissage-Insertion soulignait que pendant une génération, on a convaincu les personnes ayant une déficience auditive qu'elles pourraient faire de la comptabilité. Aujourd'hui, ce secteur est moins pourvoyeur d'emploi. Les choix d'apprentissage proposés par les organismes d'accompagnement peuvent se trouver décalés par rapport aux besoins des entreprises.

\subsection{Formation insuffisante des travailleurs handicapées}
Si l'on s'en réfère à la stratégie européenne pour la croissance et l'emploi dite « stratégie de Lisbonne », l'objectif des chefs d'Etat ou de gouvernement est de faire de l'Union européenne (UE) "\textit{l'économie de la connaissance la plus compétitive et la plus dynamique du monde capable d'une croissance économique durable accompagnée d'une amélioration quantitative et qualitative de l'emploi et d'une plus grande cohésion sociale}"\footnote{Conseil européen de Lisbonne 23 et 24 mars 2000, Conclusions de la Présidence,
\url{www.europarl.europa.eu/summits/lis1_fr.htm}
} d'ici 2010.\\

Les lignes directrices pour la croissance et l'emploi (2005-2008) adoptées par le Conseil de l'Union européenne en application de cette stratégie précisent notamment : " \textit{Pour favoriser l'accès à l'emploi à tout âge, augmenter les niveaux de productivité et la qualité de l'emploi, l'UE doit investir davantage et plus efficacement dans le capital humain et dans l'éducation et la formation tout au long de la vie, ce qui sera bénéfique pour les citoyens, les entreprises, l'économie et la société} ".\\

L'importance de cette question pour les personnes en situations de handicap semble faire l'objet d'un consensus politique que ce soit au plan international ou national.\\

En France aujourd'hui, la formation professionnelle des adultes reste insuffisamment accessible aux personnes ayant un handicap en raison du cloisonnement des dispositifs, du manque d'investissement sur un sujet considéré comme mineur ou technique par les acteurs sociaux qui seraient à même " \textit{d'abaisser les barrières }" (Pouvoirs publics/ syndicats de salariés/ employeurs/ associations ...) et du manque d'information des intéressés qui en découle.~\cite{accesFormationTH}


\section{Stéréotypes et Préjugés}

De nombreux stéréotypes sévissent encore dans les consciences des employeurs, ralentissant considérablement l'embauche des travailleurs handicapés. Ces stéréotypes, vecteurs de préjugés sur le handicap ont souvent porter atteinte à la productivité plus faible des travailleurs handicapées.\\

Une étude DARES~\cite{etudeDares1} pour l'accès à l'emploi des travailleurs handicapés, montre que plus des trois quarts des entreprises rencontrées ont en effet rapporté que ces craintes liées à la productivité étaient les premières exprimées par les collaborateurs quand était évoqué le recrutement d’une personne handicapée.

\begin{quotation}
\textit{La première crainte je pense, c’est le risque que la personne soit moins compétitive qu’une autre, que le handicap l’empêche de travailler correctement. Pour ne pas parler la langue de bois, on est quand même obligés de reconnaître qu’il y a une partie de vérité parce que sinon, si les personnes handicapées n’avaient aucune difficulté à tenir un poste, eh bien à la limite, elles ne seraient pas handicapées ; je veux dire, si on est handicapé, c’est bien qu’on a des difficultés.} chargée de mission Travailleurs handicapés. Témoignage pris dans le rapport DARES~\cite{etudeDares1}\\
\end{quotation}


Par ailleurs, en 2003, une enquête \footnote{Enquête complète disponible sur le site du Sénat : \url{http://www.senat.fr/rap/a05-214/a05-2142.html}} a été faite par Jean-François Amadieu, professeur à l'université Paris 1 et directeur de l'Observatoire des discriminations. L'enquête consistait à envoyer à 258 entreprises, près de 1800 candidatures qui correspondaient à 7 candidats virtuels :
\begin{enumerate}
\item Un homme dit "standard" (nom et prénom français, résident à Paris, blanc de peau, apparence standard) et considéré comme le candidat de "référence".
\item Une femme dit "standard" (nom et prénom français, résident à Paris, blanche de peau, apparence standard)
\item Un homme d'origine maghrébine (nom et prénom maghrébins, résident à Paris, apparence standard)
\item Un homme résident au Val Fourré à Mantes-la-Jolie (nom et prénom français, résident à Paris, apparence standard)
\item Un homme au visage disgracieux (nom et prénom français, résident à Paris, blanc de peau)
\item Un homme de 50 ans (nom et prénom français, résident à Paris, blanc de peau, apparence standard)
\item Un homme handicapé (nom et prénom français, résident à Paris, blanc de peau, apparence standard)
\end{enumerate}

Le candidat de référence obtint près de 75 convocations.
Et outre les quelques candidats virtuels présents pour tester d'autres discriminations (\^age, racisme, etc.), le candidat qui avait déclaré son handicap reçut 5 convocations seulement.\\

Lorsque cette étude a été refaite en 2006, les résultats furent un peu plus positifs en faveur de la personne handicapée, notamment gr\^ace à la loi de 2005.\\

Ces enqu\^etes montrent à quel point l'entreprise peine à assimiler la culture de la diversité malgré les efforts réels entrepris ces dernières années. \\
En effet, les candidats étaient à compétences égales sur les curriculum vitae. C'est donc le choix des recruteurs, motivé par des préjugés négatifs, qui s'est porté plut\^ot sur le candidat de référence. La différence est en effet ressentie en entreprise comme une prise de risque dans un environnement où on se soucie plutôt de les réduire.\\

Pour ce qui concerne le regard porté par les autres salariés, il semble marqué davantage par un sentiment de curiosité que par une appréhension particulière. La gêne de ne pas assez en savoir sur l'histoire du handicap de la personne génère une volonté d'entraide et de témoigner de sa solidarité (TEMOIGNAGES EN ANNEXE)


\section{Méconnaissance par les employeurs des mesures incitant l'embauche}
\label{mesuresIncitatives}

Différentes mesures ont été mises en place par l'Etat pour inciter les entreprises à embaucher des personnes en situation d'handicap. \\

La plus forte mesure incitative, déjà introduite dans l'introduction, est la contribution que l'entreprise doit verser à l'AGEFIPH en cas de non-respect du quota des 6 \%. Le montant de la contribution est proportionnel au nombre de personnes manquantes et varie selon le nombre total de salariés dans l'entreprise.
Cette mesure est connue de toutes les entreprises, car elle représente une perte économique pour les entreprises employant moins de 6 \% de travailleurs handicapés.\\

Si la contribution à l'Agefiph est une contrainte pour l'entreprise, il existe d'autres mesures, visant à aider l'embauche de travailleurs handicapés qui sont cependant peu connues par les entreprises.\\

Parmi ces mesures, nous pouvons distinguer d'une part les aides apportées aux entreprises telles que les contrats aidés qui peuvent apporter quelques avantages aux employeurs ou les primes uniques versées à l'employeur lors de l'embauche d'un travailleur handicapé.\\

D'un autre c\^oté, des mesures visent à réduire l'écart entre la moindre productivité \textit{supposée} de l'individu et son co\^ut du travail. Celles-ci sont plus ou moins spécifiques en fonction des personnes handicapées. En diminuant le salaire d'une personne handicapée de 10 à 20 \%, selon le profil de la personne, sa moindre productivité \textit{supposée} par rapport à celle d'un travailleur valide est alors compensée. Son co\^ut de travail étant finalement ramené à sa productivité, son recrutement est possible.

\section{R\^ole délétère ou paradoxal des allocations d'aides}
\label{allocationsAide}

En France, actuellement, plusieurs indemnités financières visent à compenser les conséquences du hanicap :
\begin{itemize}
\item Les rentes d'accidents du travail ou de maladies professionnelles
\item Les pensions d'invalidité pour les origines non professionnelles
\item L'AAH\footnote{AAH : Allocation Adulte Handicapé. Représente le minimum social pour les personnes handicapées} pour les personnes sans ressources\\
\end{itemize}

Dans la grande majorité des cas, ces aides sont versées sous condition de ressources. 
Une reprise d'emploi peut entraîner, soit la suppression, soit une forte diminution du montant de ces aides, ce qui peut représenter un frein à la mobilisation de la personne autour d'un projet professionnel. Cet effet devrait être partiellement revu par la nouvelle loi sur l'égalité des chances, la participation et la citoyenneté des personnes handicapées.

\section{Méconnaissance par le public cible des dispositifs d'accompagnement}

\subsection{Organismes d'accompagnement}
De nombreux organismes d'accompagnement existent pour faciliter l'accès à l'emploi pour les personnes handicapées : recherche d'entreprises selon les profils des personnes, insertion professionnelle, médiation pour l'installation d'aménagements de poste, maintien dans l'emploi.\\

\textbf{CapEmploi} \footnote{\url{http://www.capemploi.net/cap-emploi/}} est une des principales organisations oeuvrant pour l'emploi des personnes handicapées en France. Financée par l'Agefiph et dispersée à travers 118 agences en France, elle propose des services dont :
\begin{itemize}
\item L'évaluation et le diagnostic, c'est à dire l'identification des freins à l'emploi, des compétences, des conditions d'adéquation à l'emploi par rapport au handicap.
\item L'élaboration et la validation d'un projet professionnel
\item L'accès à la formation
\item L'appui à la recherche d'emploi
\item L'appui à l'embauche\\
\end{itemize}

Créée le 11 décembre 1992, l'association \textbf{Comete France} œuvre pour le maintien d'une dynamique d'insertion sociale et professionnelle, pour, autour et avec, les personnes hospitalisées dans les 41 établissements adhérents. L'action de Comete France vise à développer des stratégies précoces d'insertion sociale et professionnelle permettant de construire, dès l'entrée de la personne dans un établissement ou service de Soins de Suite et de Réadaptation (SSR), spécialisé en Médecine Physique et de Réadaptation (MPR) et avec sa participation volontaire et active, un projet de vie, incluant obligatoirement une dimension professionnelle, qui pourra se concrétiser le plus rapidement possible après sa sortie de l’établissement sanitaire.

\subsection{Méconnaissance de ces organismes}
\subsubsection{Par les personnes en situation de handicap}
Il apparaît que de nombreuses personnes n'ont pas eu recours à ces associations hors de leur recherche d'emploi, parce qu'elles en ignoraient l'existence.\\

C'est encore aujourd'hui le hasard d'un adressage à une structure de médecine physique et de réadaptation, adhérent du réseau Comète France, qui permet la rencontre avec ce service. \\

La rencontre de cette équipe au centre de rééducation neurologique Propara à Montpellier m'a permis de mesurer la réactivité de la prestation offerte par cette équipe. Dès les premières semaines d'hospitalisation de la personne, après un accident ou une maladie, la question de la reprise de l'emploi est mise sur la table de la discussion et permet au patient de se projeter dans une perspective de réintégration sociaux professionnels, quelque soit son pronostic de récupération fonctionnelle. Mes interlocuteurs (Ergothérapeute et Ergonome) m'ont indiqué combien cette démarche active permettait au patient de débuter une démarche de deuil de leur vie antérieure.

\subsubsection{Par les employeurs}
CapEmploi offre, en plus des services dédiés aux personnes handicapées, des conseils et formations à destination des employeurs.
Ces services sont de 4 types :
\begin{itemize}
\item Des informations sur l'Emploi des personnes Handicapées.
\item De la sensibilisation de l'employeur et des équipes de travail à l'emploi des personnes handicapées (notamment pour faciliter l'accueil et l'intégration des personnes handicapées dans l'équipe et l'environnement de travail).
\item Des conseils pour la préparation et le recrutement de personnes handicapées
\item De l'aide au recrutement de personnes handicapées.\\
\end{itemize}

Cependant, ces dispositifs sont souvent méconnus de la plupart des entreprises. 
Et dans le cas où les entreprises connaissent le réseau, toutes ne l'exploitent pas de la même façon. Certaines, par exemple, sont découragées devant le nombre important d'acteurs facilitant l'insertion professionnelle des personnes handicapées et ne parviennent pas à identifier les interlocuteurs susceptibles de les aider. D'autres attendent que ces associations les contactent pour leur proposer des actions ou des profils de candidats.

\subsubsection{Par les médecins du travail}
Nombreuses sont les personnes en situation de handicap qui, faute d'une communication précoce entre le médecin du travail et l'équipe de rééducation qui les soigne, se plaignent d'être considérées trop facilement comme inapte à reprendre un emploi ou à bénéficier d'un reclassement professionnel.


