\chapter{Méthodologie}

Pour appuyer mes résultats et ma réflexion, j'ai décidé d'orienter mes recherches vers plusieurs sources d'informations.

\section{Recherche bibliographique}
La recherche bibliographique est la première source d'information qui m'a paru primordiale pour bien analyser les notions liées au handicap. Toutes les références et informations se basent sur des données provenant de sites officiels (le site de l'AGEFIPH par exemple). Les mots clés utilisés (\# handicap, \# insertion professionnelle, etc.) m'ont permis de sélectionner des articles pertinents et de bonne qualité.

\section{Rencontre de deux patients et témoignages}
Après avoir cerné les principales idées qui répondront à mes problématiques, j'ai pu avoir l'opportunité d'en discuter avec deux patients du centre de rééducation fonctionnelle Propara, à Montpellier.

\section{Rencontre d'un département COMETE}
Enfin, j'ai pu rencontrer deux personnes (une ergothérapeute et une ergonome) issues de la cellule Comète France (COMmunication Environnement Tremplin vers l'Emploi). L'association Comète agit, à travers des équipes dédiées, dans une quinzaine de centres de rééducation en France. Leur objectif : veiller à ce que la personne, dès la phase de soins, soit dans une dynamique de retour à l'emploi et assister l'entreprise pour préparer le retour du salarié.

\section{Évaluation sur le terrain}
Mon expérience dans l'association HandiManagement m'a permis de rencontrer des profils différents : personnes en situation d'handicap et travaillant en entreprise, gérants d'association pour la réinsertion professionnelle de personnes handicapées, responsables de mission handicap, médecins.