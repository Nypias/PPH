\chapter{Méthodologie}

La recherche d'informations s'est appuyée sur plusieurs sources. Loin de prétendre à une quelconque exhaustivité, elle s'est inscrite dans le souci de concilier des données livresques et des données vécues et partagées avec des personnes handicapées et des professionnels de l'insertion professionnelle.

\section{Recherche bibliographique}
La recherche bibliographique pour un sujet aussi vaste ne pouvait prétendre faire un appel à un moteur de recherche commme Google. En effet, le croisement de 2 mots clés tels que Handicap et Insertion professionnelle amène à la découverte de 1,46 millions de liens relatifs à ce sujet. 
Par souci d'accéder à une information pertinente, ce sont 3 sites officiels qui ont fait l'objet d'un inventaire bibliographique: 

\begin{itemize}
\item le site de Légifrance
\item le site de l'Agefiph (Association de gestion du fonds pour l'insertion professionnelle des personnes handicapées)
\item le site de l'association Comète France (Communication Environnement Tremplin vers l'Emploi)\\
\end{itemize}


Le 1er site rassemble toute la législation sur la santé et le travail parmi d'autres sujets.\\

Le 2ème site concerne le principal partenaire de la politique de l'emploi des personnes handicapées mené par les pouvoirs publics. L'Agefiph est l'association chargée de gérer les contributions financières versées par les entreprises de 20 salariés et plus, soumises à l'obligation d'emploi des personnes handicapées. Au service des personnes handicapées et des entreprises, l'Agefiph propose des aides financières et des services pour les personnes handicapées, les entreprises et pour certaines associations choisies, comme l'association Comète France.\\

Le 3ème site est celui du seul réseau européen d'établissements de médecine physique et de réadaptation, qui s'est imposé une démarche précoce d'insertion socioprofessionnelle. Ce sont aujourd'hui près de 40 structures françaises organisées en équipes multidisciplinaires (médecins, ergothérapeutes, ergonomes, assistants de service social, chargés d'insertion, psychologues). Les départements Comète offrent aux patients qui le désirent un accompagnement dans la recherche d'emploi, l'insertion professionnelle et le maintien dans l'emploi. Pour pouvoir bénéficier des financements de fonctionnement (assuré par l'Agefiph et l'Agence Régionale de Santé), il faut que l'équipe Comète honore une obligation de résultats (nombre de personnes par an).\\

Dans chacun des sites, seules les informations référencées (date, auteur et contenu) ont été retenues.

\section{Rencontre de deux patients et témoignages}
Un recueil de témoignages de personnes handicapées motrices a été réalisé dans un centre de Médecine Physique et de Réadaptation spécialisé dans la prise en charge de patients victimes d'accidents de la voie publique responsable d'une lésion médullaire. Ce centre situé à Montpellier accueille surtout des personnes paraplégiques et tétraplégiques.\\
Un questionnaire standardisé leur a été livré. 

\section{Rencontre d'un département COMETE}
La rencontre d'une équipe de Comète France s'est imposée pour m'offrir un retour d'expérience sur les difficultés rencontrées par les professionnels de l'insertion professionnelle dans l'accompagnement des personnes handicapées. 
Cette équipe était composé d'une ergothérapeute dont la mission principale est d'aider les personnes handicapées à retrouver leur autonomie ; et d'une ergonome qui aident à la conception et à l'amélioration des postes de travail afin que les personnes handicapées puissent travailler dans les meilleurs conditions possibles, avec le maximum de confort, de sécurité et d'efficacité.
Le recueil d'informations s'appuie sur un questionnaire ouvert et une discussion informelle.

\section{Évaluation sur le terrain}
Enfin, la rencontre des interlocuteurs sur le terrain de l'emploi s'est imposé comme le dernier univers dans lequel il m'a fallu me plonger pour mieux percevoir les réalités partagées entre les différents acteurs. En effet, mon expérience dans l'association HandiManagement m'a permis de rencontrer des profils différents : personnes en situation d'handicap et travaillant en entreprise, gérants d'association pour la réinsertion professionnelle de personnes handicapées, responsables de mission handicap, médecins.