\chapter{Discussion}

Nous nous proposons maintenant de discuter autour de deux questions centrales à l'aide des résultats observés en chapitre 3.

\section{Pourquoi encourager l'emploi des personnes handicapées en pleine conjoncture défavorable ?}

Actuellement, l'emploi n'est pas à son point fort. Quand le taux de chômage pour les personnes non handicapées se positionne à hauteur de 9 \%, le taux de chômage des personnes handicapées est doublé.\\

L'insertion professionnelle des personnes handicapées est très importante pour plusieurs raisons.

\subsection{Arguments sociaux}
Pour lutter contre la solitude qui sévit pour les personnes handicapées, il est nécessaire de retrouver une vie sociale épanouie. La réinsertion professionnelle permet ce retour grâce aux rencontres et échanges avec les collaborateurs de travail. \\
On retrouve souvent le cas de la personne qui travaillait avant d'avoir un accident handicapant et qui souhaite retrouver sa faculté à travailler, à avoir des échanges.

\subsection{Arguments économiques}
Les arguments économiques sont valables pour le patient et pour l'employeur.
\subsubsection{Pour le patient}
D'une part, le patient pourra regagner de l'argent car son salaire sera supérieur aux allocations d'aides versées. En travaillant, il pourra aussi recommencer à économiser pour la retraite (PRISE EN CHARGE DES RETRAITES EN TEMPS NORMAL ?) 

\subsubsection{Pour l'employeur}
En employant des personnes en situation de handicap, l'employeur voit son nombre de personnes handicapées augmenter dans son entreprise et ainsi, la taxe versée à l'AGEFIPH diminuer.

\subsection{Arguments psychologiques}
Auto valorisation ET Statut grandissant dans la famille


\section{Comment lutter contre les facteurs de résistance à la reprise de l'emploi ?}

\subsection{Du c\^oté de l'employeur}

\subsubsection{Rappeler le dispositif de législation}

\subsubsection{Sensibiliser les employeurs sur la force productive des personnes handicapées}

Lever les préjugés !\\

Un stéréotype est une idée, une opinion, acceptée sans réflexion et répétée, sans avoir été soumise à un examen critique, par un personne ou par un groupe, et qui détermine, à un degré plus ou moins élevé, ses manières de penser, de sentir et d'agir.
La conséquence des stéréotypes est une généralisation abusive d'une idée. 
Parmi un témoignage, un chef d'une entreprise industrielle du secteur de l'énergie pensait que le recrutement de personnes handicapées n'était pas envisageable parce que son bâtiment ne possédait pas d'escaliers. Il partait donc sur le stéréotype qu'une personne handicapée est forcément en fauteuil roulant. Or parmi l'ensemble des personnes handicapées, près de 3 \% sont effectivement en fauteuil roulant. Ce responsable avait donc développé un stéréotype basé sur le fait qu'il ne pourrait jamais embaucher de personnes handicapées. Alors qu'il pouvait embaucher près de 97 \% de personnes handicapées pour lesquelles prendre les escaliers n'était pas un problème.\\

Plus généralement, on remarque que le stéréotype le plus commun pour le milieu du handicap est de croire que le handicap est toujours visible. Or, près de 80 \% des personnes handicapées ont un handicap invisible. Les handicaps psychiques (dépression, schyzophrénie, etc.), cognitifs (dyslexie...), mentaux ou sensoriels sont majoritairement des handicaps invisibles.\\

Un préjugé signifie "juger avant". Il est défini comme une opinion hâtive et préconçue souvent imposée par le milieu, l'époque, l'éducation ou due à la généralisation d'une expérience personnelle ou d'un cas particulier. Par exemple, nos comportements envers une personne dépendent souvent de ce que nous imaginons de elle, et pas de ce qu'elle est réellement.\\
Les effets des stéréotypes et de la généralisation étant souvent plus importants sur les populations minoritaires, si le recrutement d'une personne handicapée se solde par un échec, les stéréotypes négatifs risquent d'être renforcés. A contrario, si le recrutement d'une personne non handicapée se déroule mal, alors le phénomène de généralisation n'aura pas lieu.\\
Pour prendre un exemple, on entend souvent des chefs d'entreprise dire que si l'embauche d'une personne handicapée s'est mal passée, alors le chef d'entreprise ne reprendra plus d'autres personnes handicapées. Alors qu'avec des personnes non handicapées, cette réflexion ne sera jamais entendu.\\

En 2003, une enquête a été faite par Jean-François Amadieu, professeur à l'université Paris 1 et directeur de l'Observatoire des discriminations. L'enquête consistait à envoyer à 258 entreprises, près de 1800 candidatures qui correspondaient à 7 candidats virtuels :
\begin{enumerate}
\item Un homme dit "standard" (nom et prénom français, résident à Paris, blanc de peau, apparence standard) et considéré comme le candidat de "référence".
\item Une femme dit "standard" (nom et prénom français, résident à Paris, blanche de peau, apparence standard)
\item Un homme d'origine maghrébine (nom et prénom maghrébins, résident à Paris, apparence standard)
\item Un homme résident au Val Fourré à Mantes-la-Jolie (nom et prénom français, résident à Paris, apparence standard)
\item Un homme au visage disgracieux (nom et prénom français, résident à Paris, blanc de peau)
\item Un homme de 50 ans (nom et prénom français, résident à Paris, blanc de peau, apparence standard)
\item Un homme handicapé (nom et prénom français, résident à Paris, blanc de peau, apparence standard)
\end{enumerate}

Le candidat de référence obtint près de 75 convocations.
Et outre les quelques candidats virtuels présents pour tester d'autres discriminations (\^age, racisme, etc.), le candidat qui avait déclaré son handicap reçut 5 convocations seulement.\\

Lorsque cette étude a été refaite en 2006, les résultats furent un peu plus positifs en faveur de la personne handicapée, notamment gr\^ace à la loi de 2005.\\

Ces enqu\^etes montrent à quel point l'entreprise peine à assimiler la culture de la diversité malgré les efforts réels entrepris ces dernières années. \\
En effet, les candidats étaient à compétences égales sur les curriculum vitae. C'est donc le choix des recruteurs, motivé par des préjugés négatifs, qui s'est porté plut\^ot sur le candidat de référence. La différence est en effet ressentie en entreprise comme une prise de risque dans un environnement où on se soucie plutôt de les réduire.

\subsubsection{Accompagner les employeurs}
Grâce à des professionnels : ergothérapeutes, ergonomes, chargé de mission, psychologue ou médecins.

\subsubsection{Établir une passerelle}
Entre le médecin du travail, les associations d'aide à l'insertion professionnelle et au maintien dans l'emploi (par exemple Comète) et avec l'employeur.

\subsection{Du c\^oté de la personne handicapée}

\subsubsection{Rappeler les arguments et la valeur ajoutée de la reprise du travail}

\subsubsection{Agir très précocement après une maladie ou un accident}
Pour ne pas prolonger et pérenniser l'arrêt.
Si rééducation fonctionnelle, nécessaire, ouverture d'un dossier et établissement de contacts avec l'employé.